\documentclass{article}
\usepackage[utf8]{inputenc}
\usepackage{latexsym}
\usepackage{amsmath,amssymb,amsthm}
\usepackage{amsfonts}
\usepackage{enumerate}
\usepackage{verbatim}
\usepackage[document]{ragged2e}
\usepackage{indentfirst}

\title{Document for class 3} 
\author{Rashad Mahmudov}
\date{March 16, 2020}

\begin{document}

\maketitle

\section{Introduction}
\small{\justify{Curves and surfaces are usually defined with a minimum assumption like differ-entiability or rectifiability.This hypothesis seems to be natural and are useful to access to some basic properties such that the possibility to compute the length of a curve $C(t)$ between to point $C(a)$ and $C(b) : L = \int_a^b || C'(t||dt$. But what arise if these hypothesis are not verified? Is is possible to defined curves which
are no-where differentiable? Is it possible to control they geometry? The fractal theory gives numerous answers of these questions.

\setlength{\parindent}{0.5cm} This chapter is not devoted to fractal geometry and it don’t give answer
to all of these questions. But it describes how it is possible to use the fractal
paradigm to design a set of curves which are not possible to describe with others
models because of the simple fact that these models made the assumption of the differentiability.}}

\section{Some math}
In this section we gives some mathematics results used in this course. Profs can
be found.
\normalsize{\paragraph{Definition 1} (Contraction). Let $(\chi,d)$ be a metric space. Let $f$ be a trans-
formation on $(\chi,d). \,\,f$ is a contraction $(\chi,d)$  iff 
$\exists s \in \mathbb{R},\, s < 1$ tq $\forall x,y \in$

$\chi,d(f(x), f(y)) \leq s\times d(x, y)$ 
\paragraph{Definition 2}(Banach Fixed Point theorem).Let $(\chi,d)$ be a non-empty com-
plete metric space and $f$ a contraction on $(\chi,d)$. Then, there exists a unique
point $x \in \chi$,named the fixed point of $f$, verifying $f(x) = x$.}
\section{Lists example}
\begin{enumerate}
  \item First entry
  \item Second entry
  \item Third entry
\end{enumerate}
\newpage
\begin{itemize}
  \item First entry
  \item Second entry
  \item Third entry
\end{itemize}

\subsection{Package enumerate}
\justify{This package gives the enumerate environment an optional argument which
determines the style in which the counter is printed. An occurence of one of the tokens A a I i or 1 produces the value of the counter printed with (respectively)

\noindent\verb|\Alph \alph \Roman \roman or\arabic|. These letters may be surrounded by any strings involving any other TEX expressions, however the tokens A a I i 1
must be inside a group if they are not to be taken as special.

\indent You have to load package \verb|enumerate| writing \verb|\usepackage{enumerate}|.}

\newcommand{\myparagraph}{\mbox{}\\}
\paragraph{Examples}\myparagraph

EX i. one one one one one one one one one one one\myparagraph

\hspace{-1mm}EX ii. two\myparagraph

example a) one of two one of two one of two \myparagraph

example b) two of two \myparagraph

\paragraph{Remark}\myparagraph\vspace{-5mm}

For writing \LaTeX\, code inside \LaTeX\, document you have to load and use
\verb|verbatim| package.

\end{document}
