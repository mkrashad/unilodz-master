\documentclass{article}
\usepackage[utf8]{inputenc}
\usepackage{sectsty}
\usepackage{color}
\usepackage{graphicx}

\title{More packages – \textbf{graphics} 

Class 5 (26-03-20)}
\author{Rashad Mahmudov}
\date{}

\begin{document}

\maketitle
\section*{Colours graphics in the document}
First you have to input packages using
\verb|\usepackage{color}and \usepackage{graphicx}|.
Then read the packages documentation.

This document ”Packages in the graphics bundle” serves as a user-manual
for the packages \verb|color, graphics, and graphicx|. Further documentation may
be obtained by processing the source (dtx) files of the individual packages.

\subsection*{Package \normalfont{colour}}
One special option for the \verb|color| package that is of interest is \verb|monochrome|. If
this option is selected the colour commands are all disabled so that they do not
generate errors, but do not generate colour either. This is useful if previewing
with a previewer that can not produce colour.

\subsection*{Defining colours}
The colours \verb|black,white,red,green,blue,cyan,magenta,yellow |should be predefined, but should you wish to mix your own colours use the \verb|\definecolor| command.

\begin{verbatim}
\definecolor{light-blue}{rgb}{0.8,0.85,1}
\definecolor{mygrey}{gray}{0.75}
\end{verbatim}

To use predefined colour you have to write \verb|\color{name}|. To write text in colour you write \verb|\textcolor{name}{text}|.

\textcolor{red}{Users are encouraged to produce different versions of this file for any printers they use.}

\textcolor{blue}{Users are encouraged to produce different versions of this file for any printers they use.}

\textcolor{magenta}{Users are encouraged to produce different versions of this file for any printers they use.}

\newpage

\section*{Package \normalfont{graphicx}}

Package \verb|graphicx| is ‘extended’ textttgraphics package.

Usage: \verb|\includegraphics[options]{file.jpg}|. It is possible to use also
.pdf and .png files.

As \verb|options| you can write for example: scale=number, rotate=angle, width=value,
heigth=value.

\subsection*{Some examples}


\begin{enumerate}
  \item Picture in normal size (here we use option scale=0.1).
  \item[] \includegraphics[scale=0.1]{logo.jpg} \hspace{2mm} \includegraphics[scale=0.1]{logo-1.jpg}
  \item Picture rotated. We use option angle=45 and -45.
  \item[] \includegraphics[scale=0.1,angle=45]{logo-1.jpg} \hspace{2mm} \includegraphics[scale=0.1,angle=-45]{logo-1.jpg}
\end{enumerate}

















\end{document}
